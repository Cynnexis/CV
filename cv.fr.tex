% LaTeX engine: XeLaTeX
\documentclass{resume}

\usepackage[english]{babel}
\usepackage{verbatim}
\usepackage{float}
\usepackage{graphicx}
\usepackage[table]{xcolor}
\usepackage{csquotes}
\usepackage[top=0cm, bottom=0cm, left=0cm, right=0cm, headsep=0cm]{geometry}
\usepackage{hyperref}

\graphicspath{{./images/}}

\begin{document}

\noindent\fcolorbox{primary-color}{primary-color}{
	\begin{minipage}{\paperwidth}
		\vspace{1.2cm}
		\myname
		
		\mytitle{Recherche un stage en tant qu’Ingénieur\\Informatique}
		\vspace{1.2cm}
	\end{minipage}
}

\noindent\fcolorbox{margin-background-color}{margin-background-color}{
	\begin{minipage}[t][0.83\textheight]{\firstcolumnwidth}
		\jump[1]
		\begin{adjustwidth}{0.3cm}{0cm}
			\firstwindowicon{person}{
				\begin{adjustwidth}{0.3cm}{0cm}
					\grouptitle{Contact}
				\end{adjustwidth}
			}
			
			\jump
			\firstwindowicon{location}{
				\entrytitle{Adresse}
				\entrydescription{63 Chemin du Moulin, 38660, La Terrasse, FRANCE}
			}
			
			\bigjump
			\firstwindowicon{phone}{
				\entrytitle{Numéro de Téléphone}
				\entrydescription{\link{tel:+33751648450}{+33 7 51 64 84 50}}
			}
			
			\bigjump
			\firstwindowicon{email}{
				\entrytitle{Email}
				\entrydescription{\link{mailto:Valentin.berger38@gmail.com}{Valentin.berger38@gmail.com}}
			}
			
			\bigjump
			\firstwindowicon{linkedin}{
				\entrytitle{LinkedIn}
				\entrydescription{\link{https://www.linkedin.com/in/valentin-berger-075800155/}{https://www.linkedin.com/in}\\\link{https://www.linkedin.com/in/valentin-berger-075800155/}{/valentin-berger-075800155/}}
			}
			
			\bigjump
			\firstwindowicon{skype}{
				\entrytitle{Skype}
				\entrydescription{valentin.berger}
			}
			
			\bigjump
			\firstwindowicon{github}{
				\entrytitle{GitHub}
				\entrydescription{\link{https://github.com/Cynnexis}{https://github.com/Cynnexis}}
			}
			
			\Bigjump
			\firstwindowicon{language}{
				\begin{adjustwidth}{0.3cm}{0cm}
					\grouptitle{Langues}
				\end{adjustwidth}
			}
			
			\jump
			\firstwindowicon{flag-fr}{
				\entrytitle{Français}
				\entrydescription{Langue Natale}
			}
			
			\jump
			\firstwindowicon{flag-ca}{
				\entrytitle{Anglais}
				\entrydescription{Bonne maîtrise, Niveau B2 (CEFR),}
				\entrydescription{6.5/9 à l'IELTS}
			}
			
			\jump
			\firstwindowicon{flag-es}{
				\entrytitle{Espagnol}
				\entrydescription{Niveau débutant}
			}
			\vfill
		\end{adjustwidth}
	\end{minipage}
}%
\hfill
\begin{minipage}[t][0.83\textheight]{\secondcolumnwidth}
	\jump[1]
	\secondwindowicon{school}{
		\grouptitle{Diplômes}
		
		\entrytitle{Cycle Ingénieur à Polytech Lyon en Informatique}
		\entrydate{2017 -- Année actuelle}
		\entrydescription{Polytech Lyon, quatrième année du cycle ingénieur}
		
		\jump
		\entrytitle{L1 et L2 Mathématiques-Informatique}
		\entrydate{2015 -- 2017}
		\entrydescription{Université Grenoble Alpes}
	}
	
	\bigjump
	\secondwindowicon{work}{
		\grouptitle{Expériences Professionnelles}
		
		\entrytitle{McMaster University \& TOTAL}
		\entrydescription{TOTAL est une entreprise française de pétrol. McMaster est une université située à Hamilton, au Canada.}
		\entrydate{Septembre 2018 jusqu'à Janvier 2019 -- 5 mois}
		\entrydescription{
			Stage de recherche en Optimization Mathématique.
			La mission était de manipuler une formulation d'un model mathématique sur une raffinerie de façon à gérer les arrêts non-prévus des machines, avec différentes durée d'arrêt.
			
			\begin{itemize}
				\item Formuler un model mathématique multi-périod
				\item Implentation d'un GUI
			\end{itemize}
		}
		
		\jump
		\entrytitle{STMicroelectronics Crolles}
		\entrydescription{Fabricant de semi-conducteurs}
		\entrydate{Juillet et Août 2017 -- 2 mois}
		\entrydescription{
			Intérimaire en tant que opérateur en salle blanche
			
			\begin{itemize}
				\item J’ai pris soin des machines dans la section Lithographie
			\end{itemize}
		}
		\entrydate{Juin 2016 -- 1 mois}
		\entrydescription{
			Stage dans une entreprise centrée sur la fabrication de puce électronique
			
			\begin{itemize}
				\item J’ai développer une application en C\#
				\item J’ai travaillé dans une équipe d’ingénieurs et de techniciens spécialisés en Informatique
			\end{itemize}
		}
		
		\jump
		\entrytitle{Green Dog Studio}
		\entrydescription{Une PME spécialisée dans le développement de sites web}
		\entrydate{Juin 2013 -- 1 semaine}
		\entrydescription{
			Stage dans une PME spécialisée dans le Web Design
			
			\begin{itemize}
				\item J’ai crée un site web en HTML5 et CSS3
			\end{itemize}
		}
	}
	
	\bigjump
	\secondwindowicon{poll}{
		\grouptitle{Compétences}
		
		\entrytitle{Systèmes d’Exploitation}
		\entrydescription{Systèmes UNIX (en particulier Linux), Windows XP, 7, 8 et 10}
		
		\jump
		\entrytitle{Logiciels}
		\entrydescription{Maîtrise de Microsoft Office, Libreoffice, OpenOffice et Google Docs}
		\entrydescription{Bonne maîtrise de Visual Studio 2015/2017, Eclipse, Android Studio et IntelliJ}
		
		\jump
		\entrytitle{Langages de Programmation, Scriptes et Frameworks}
		\entrydescription{
			\begin{itemize}
				\item C, C++, C\#, Java, Python, OCaml, Ada et LaTeX
				\item HTML5, CCS3, JavaScript, PHP, OracleSQL, R, Matlab et GAMS
				\item .NET Framework, Laravel (Framework similaire Symphony)
			\end{itemize}
		}
	}
	
	\vfill
	\hfill{\scriptsize\textcolor{muted-color}{Ce document a été compilé avec {\DejaSans ❤️} par {\fontfamily{lmr}\selectfont\LaTeX{}}}} $\quad$
	
\end{minipage}

\end{document}
