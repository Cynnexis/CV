% LaTeX engine: XeLaTeX
\documentclass{resume}

\begin{document}

\header{Recherche un stage en tant qu’Ingénieur\\Informatique, en Intelligence Artificielle}

\sidepanel{
	\contact{Contact}{
		\iconentry{location}{Adresse}{63 Chemin du Moulin, 38660, La Terrasse, FRANCE}

		\bigjump
		\iconentry{phone}{Numéro de Téléphone}{\link{tel:+33751648450}{+33 7 51 64 84 50}}

		\bigjump
		\iconentry{email}{Email}{\link{mailto:Valentin.berger38@gmail.com}{Valentin.berger38@gmail.com}}

		\bigjump
		\iconentry{linkedin}{LinkedIn}{\link{https://www.linkedin.com/in/valentinberger/}{linkedin.com/in/valentinberger}}

		\bigjump
		\iconentry{skype}{Skype}{valentin.berger}

		\bigjump
		\iconentry{github}{GitHub}{\link{https://github.com/Cynnexis}{https://github.com/Cynnexis}}
	}

	\languages{Langues}{
		\iconentry{flag-fr}{Français}{Langue natale}

		\bigjump
		\iconentry{flag-ca}{Anglais}{Bonne maîtrise, Niveau C1 (CEFR),\\TOEIC 955/990}

		\bigjump
		\iconentry{flag-es}{Espagnol}{Niveau débutant}
	}

	\interests{Interêt}{
		\firstwindowicon{running}{\entrydescription{Course à  pied}}

		\jump
		\firstwindowicon{write}{\entrydescription{\'{E}criture}}

		\jump
		\firstwindowicon{computer}{\entrydescription{Programmation}}

		\jump
		\firstwindowicon{space}{\entrydescription{Espace}}
	}
	\vfill
}%
\hfill
\mainpanel{
	\displomas{Diplômes}{
		\entrytitle{Master en Intelligence Artificielle}
		\entrydate{2019 -- Année actuelle}
		\entrydescription{Université Lyon 1, double-diplôme avec Polytech Lyon}
		
		\jump
		\entrytitle{Cycle Ingénieur à Polytech Lyon en Informatique}
		\entrydate{2017 -- Année actuelle}
		\entrydescription{Polytech Lyon, cinquième année du cycle ingénieur}

		\jump
		\entrytitle{L1 et L2 Mathématiques-Informatique}
		\entrydate{2015 -- 2017}
		\entrydescription{Université Grenoble Alpes, classe préparatoire avec Polytech Grenoble}
	}

	\work{Expériences Professionnelles}{
		\entrytitle{McMaster University \& TOTAL}
		\entrydescription{McMaster est une université située à Hamilton, au Canada. TOTAL est une entreprise française de pétrol.}
		\entrydate{Septembre 2018 jusqu'à Janvier 2019 -- 5 mois}
		\entrydescription{
			Stage de recherche en Optimisation Mathématique.
			La mission était de manipuler une formulation d'un model mathématique sur une raffinerie de façon à gérer les arrêts non-anticipés des machines, avec différentes durée d'arrêt.

			\begin{itemize}
				\item Formulation d'un model mathématique multi-périodes,
				\item Comprendre le fonctionnement d'un optimiseur mathématique,
				\item Participation à l'écriture d'un article scientifique sur le projet,
				\item Implémentation d'un GUI en HTML, CSS, PHP et JavaScript.
			\end{itemize}
		}

		\jump
		\entrytitle{STMicroelectronics Crolles}
		\entrydescription{Fabricant de semi-conducteurs}
		\entrydate{Juillet et Août 2017 -- 2 mois}
		\entrydescription{
			Intérimaire en tant qu'opérateur en salle blanche

			\begin{itemize}
				\item Entretien des machines dans la section Lithographie.
			\end{itemize}
		}
		\entrydate{Juin 2016 -- 1 mois}
		\entrydescription{
			Stage dans une entreprise centrée sur la fabrication de puce électronique

			\begin{itemize}
				\item Utilisation d'un modèle statistique pour contrôler la qualité de produits sur la chaîne de production,
				\item Developpement d'une application en C\# avec le Framework .NET.
			\end{itemize}
		}
	}

	\skills{Compétences}{
		\entrytitle{Systèmes d’Exploitation}
		\entrydescription{Linux (en particulier Ubuntu), Windows XP, 7, 8 et 10}

		\jump
		\entrytitle{Logiciels}
		\entrydescription{Maîtrise de Microsoft Office, Libreoffice, OpenOffice et Google Docs}
		\entrydescription{Bonne maîtrise de Visual Studio 2019, Eclipse, Android Studio, IntelliJ et git}

		\jump
		\entrytitle{Langages de Programmation, Scriptes et Frameworks}
		\entrydescription{
			\begin{itemize}
				\item C, C++, C\#, Java, Python et LaTeX
				\item HTML5, CCS3, JavaScript, PHP, OracleSQL, R, Matlab et GAMS
				\item Maîtrise élémentaire de Qt (Framework C++), VueJS et Firebase
			\end{itemize}
		}
	}

	%\cvfootnote{Ce document a été compilé avec {\DejaSans ❤️} par {\fontfamily{lmr}\selectfont\LaTeX{}}}
}

\end{document}
