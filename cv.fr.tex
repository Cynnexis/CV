% LaTeX engine: XeLaTeX
\documentclass{resume}

\begin{document}

\header{Ingénieur Informatique\\assistant de recherche}

\sidepanel{
	\contact{Contact}{
		\iconentry{location}{Adresse}{1 Impasse Pommier, 69003, Lyon}

		\bigjump
		\iconentry{phone}{Numéro de Téléphone}{\link{tel:+33751648450}{+33 7 51 64 84 50}}

		\bigjump
		\iconentry{email}{Email}{\link{mailto:Valentin.berger38@gmail.com}{Valentin.berger38@gmail.com}}

		\bigjump
		\iconentry{linkedin}{LinkedIn}{\link{https://www.linkedin.com/in/valentinberger/}{linkedin.com/in/valentinberger}}

		\bigjump
		\iconentry{skype}{Skype}{valentin.berger}

		\bigjump
		\iconentry{github}{GitHub}{\link{https://github.com/Cynnexis}{https://github.com/Cynnexis}}
	}

	\languages{Langues}{
		\iconentry{flag-fr}{Français}{Langue natale}

		\bigjump
		\iconentry{flag-ca}{Anglais}{Bonne maîtrise, Niveau C1 (CEFR),\\TOEIC 955/990}

		\bigjump
		\iconentry{flag-es}{Espagnol}{Niveau débutant}
	}

	\interests{Intérêt}{
		\firstwindowicon{running}{\entrydescription{Course à  pied}}

		\jump
		\firstwindowicon{write}{\entrydescription{\'{E}criture}}

		\jump
		\firstwindowicon{computer}{\entrydescription{Programmation}}
	}
	\vfill
}%
\hfill
\mainpanel{
	\displomas{Dipl\^{o}mes}{
		\entrytitle{Master en Intelligence Artificielle}
		\entrydate{2019 -- 2020}
		\entrydescription{Université Lyon 1, double-dipl\^{o}me avec Polytech Lyon}
		
		\jump
		\entrytitle{Dipl\^{o}me d'Ingénieur à Polytech Lyon en Informatique}
		\entrydate{2017 -- 2020}
		\entrydescription{Polytech Lyon, cycle ingénieur}

		\jump
		\entrytitle{L1 et L2 Mathématiques-Informatique}
		\entrydate{2015 -- 2017}
		\entrydescription{Université Grenoble Alpes, classe préparatoire avec Polytech Grenoble}
	}

	\work{Expériences Professionnelles}{
		\entrytitle{CNRS - Institut des Sciences Cognitives Marc Jeannerod}
		\entrydate{Novembre 2020 -- présent}
		\entrydescription{Ingénieur assistant de recherche.

			\begin{itemize}
				\item Création d'une architecture Web et d'une application mobile en Flutter pour un sondage interactif,
				\item Traitement de nuages de points 3D.
			\end{itemize}
		}

		\jump
		\entrytitle{Solutec Lyon}
		\entrydescription{ESN dans le conseil numérique}
		\entrydate{Mars 2020 jusqu'à Juillet 2020 -- 5 mois}
		\entrydescription{
			Stage en Recherche \& Développement. La mission était de créer un chatbot pour répondre aux éventuelles questions administratives que les employés peuvent avoir. 

			\begin{itemize}
				\item Traitement du langage naturel avec spaCy,
				\item Développement de réseaux de neurones denses, convolutifs et récursifs (TensorFlow),
				\item Mise en place d'une base de données MongoDB sur Docker,
				\item Développement d'une Pipeline CI/CD sur GitLab.
			\end{itemize}
		}

		\jump
		\entrytitle{McMaster University}
		\entrydescription{McMaster est une université située à Hamilton, au Canada.}
		\entrydate{Septembre 2018 jusqu'à Janvier 2019 -- 5 mois}
		\entrydescription{
			Stage de recherche en Optimisation Mathématique.
			La mission était de manipuler une formulation d'un modèle mathématique sur une raffinerie de façon à gérer les arrêts non-anticipés des machines, avec différentes durées d'arrêt.

			\begin{itemize}
				\item Formulation d'un modèle mathématique multi-périodes,
				\item Comprendre le fonctionnement d'un optimiseur mathématique,
				\item Participation à l'écriture d'un article scientifique sur le projet.
			\end{itemize}
		}

		\jump
		\entrytitle{STMicroelectronics Crolles}
		\entrydescription{Fabricant de semi-conducteurs}
		\entrydate{Juillet et Août 2017 -- 2 mois}
		\entrydescription{
			Intérimaire en tant qu'opérateur en salle blanche

			\begin{itemize}
				\item Entretien des machines dans la section Lithographie.
			\end{itemize}
		}
	}

	\skills{Compétences}{
		\entrytitle{Langages de Programmation, Scriptes et Frameworks}
		\entrydescription{
			\begin{itemize}
				\item C, C++, C\#, Java, Kotlin, Dart, Python et LaTeX
				\item HTML5, CCS3, JavaScript, OracleSQL, R, Matlab et GAMS
				\item Bonne maîtrise d'Android Natif, TensorFlow et Flutter
			\end{itemize}
		}
	}

	%\cvfootnote{Ce document a été compilé avec {\DejaSans ❤️} par {\fontfamily{lmr}\selectfont\LaTeX{}}}
}

\end{document}
